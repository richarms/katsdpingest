\chapter{Configuration and Scheduling Software}

Depending upon the objectives of the experiment, DADA configuration
may change from observation to observation.

\section{Instrumental Configuration}

Based upon the parameters of observation, the {\tt dada} control
program will set up various operational parameters, including:
\begin{itemize}
\item buffer sizes: DMA, Data Block, file and network I/O
\item number of Primary nodes
\item number of Secondary nodes
\item assignment of Secondary to Primary nodes
\item operational mode: offline, simultaneous, or diskless
\item overlap required between Secondary nodes
\end{itemize}
These operational parameters will be stored in a {\bf configuration
database}, which will specify the instrument configuration and data
reduction requirements for various combinations of source, receiver,
centre frequency, band width, etc.

Entries in the configuration database may have an expiration date
associated with them.  In this manner, the observer may specify
special configuration and/or reduction options for a specific
experiment without permanently changing the default behaviour.

\section{Data Reduction}

The Configuration and Scheduling program, {\tt dadaskd}, will be used
to configure and schedule all data reduction operations.  Before the
{\bf diskless} mode of data reduction is implemented, all data will
exist as a file on either the Primary or Secondary nodes.  Whenever a
file is written to disk, an entry will be registered in a centralized
{\bf observations database}, which will contain basic header
information such as
\begin{itemize}
\item source name

\item start time (UTC)

\item centre frequency (MHz)

\item band width (MHz)
\end{itemize}
as well as the location (machine and file name) of the data.  Each
entry will also contain a time-stamped list of {\bf performed
operations}, describing when the data was written, when and how it was
processed, when it was deleted, etc.  The header information will be
used to find matches in the configuration database.  An observation
may be processed in multiple ways, as specified by the list of {\bf
requested operations} in the configuration database entry.

If it is possible to achieve two requested operations in one execution
of {\tt dspsr} then this will be done.  Otherwise, data reduction
operations will be performed one at a time.  After each operation is
completed, it will be recorded in the list of performed operations of
the observations database entry.

The scheduling software will periodically check or poll the
observations database.  Any entries that require data reduction will
be scheduled according to the data reduction parameters of the
requested operations in the corresponding configuration database
entry.  An observation will be considered completely processed when
the list of performed operations is equal to the list of requested
operations.  At this point, the raw data will be deleted or archived.
